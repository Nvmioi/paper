% ==========================
% 前置:摘要与关键词(中文/英文)
% ==========================
\chapter*{摘 要}
基于2010—2018年40个行业、1{,}599个行业配对的月度面板数据,本文在链路(行业配对)层面识别供给侧结构性去产能政策对跨行业风险溢出的影响。研究采用双重差分(DID)与事件研究相结合的统一框架,以方向性行业影响强度(II)为因变量,设置配对固定效应与时间固定效应,并在配对层进行聚类稳健推断。在平均效应层面,处理组行业相对对照组的II显著下降,表明政策对跨行业风险外溢具有相对抑制作用;在动态层面,政策实施当年效应不显著,随后逐步显现并持续存在,符合“价格—数量—信用”三通道的传导时滞;在结构性层面,基于“政策前对处理邻居的预暴露”的分组显示,高预暴露链路在政策后的抑制更强,提供了“产业链稳定器”(议价、信息、协同)机制的结构化证据。两类稳健性检验(控制变量扩展与缩尾敏感性)表明主结论在合理口径下保持稳健。

本文的贡献在于:第一,将“方向性—链路级”度量与因果识别耦合,避免宏观联动与共同趋势被误判为政策效应;第二,以政策前“预暴露”构造结构性剂量并刻画“剂量—响应”,揭示何处抑制更强并为链路治理提供依据;第三,将识别证据转译为政策工具,提出“稳链优先序—协同披露—压力测试分层”的可操作建议。研究为评估结构性产业政策在网络中的作用提供了方法参考,亦为“去产能—稳链—高质量增长”的政策组合提供证据支持。

\vspace{0.5em}
\noindent\textbf{关键词:} 去产能;行业风险溢出;双重差分;事件研究;生产网络;结构性异质性

\chapter*{Abstract}
Using monthly panel data of 40 industries and 1,599 directed industry pairs from 2010 to 2018, this study identifies the impact of China’s supply‐side capacity reduction policy on inter‐industry risk spillovers at the link (pair) level. We employ a unified framework combining Difference‐in‐Differences (DID) with an event‐study design, using the directional industry influence intensity (II) as the outcome, with pair and time fixed effects and cluster‐robust inference at the pair level. On average, the II of treated industries declines relative to controls, indicating a policy‐induced attenuation of cross‐industry spillovers. Dynamically, the effect is insignificant in the implementation year, then emerges and persists, consistent with frictions along price–quantity–credit channels. Structurally, pre‐policy exposure to treated neighbors exhibits a dose–response pattern: links with higher pre‐exposure display stronger post‐policy attenuation, consistent with a “supply‐chain stabilizer” mechanism (bargaining, information, coordination). Robustness checks on control expansion and winsorization confirm the stability of the main findings.

This paper contributes by: (i) integrating directional, link‐level measures with causal identification to avoid confounding macro comovements with policy effects; (ii) introducing pre‐policy exposure as a structural dose to reveal where attenuation is stronger and to inform link‐level governance; and (iii) translating evidence into actionable policy tools—prioritized link stabilization, coordinated disclosure, and exposure‐tiered stress testing. The results offer methodological guidance for evaluating structural industrial policies in networked settings and provide evidence for the “capacity reduction–stable chains–high‐quality growth” policy mix.

\vspace{0.5em}
\noindent\textbf{Keywords:} capacity reduction; inter‐industry spillovers; DID; event study; production networks; structural heterogeneity
