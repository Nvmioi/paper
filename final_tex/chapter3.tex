% ============================================================================
% 第3章 理论与研究设计(仅使用二级目录,不新增更深层级)
% ============================================================================
\chapter{理论与研究设计}
\label{chap:theory}

\section{理论框架}
去产能政策通过价格、数量与融资约束三条通道改变受影响行业的行为边界,并沿既有的投入—产出与合同—信用关系向外传导。上游行业的供给侧收缩,首先在价格与交付周期上体现为对下游的成本与库存压力,随后通过订单节奏、联合排产、存货周转与现金流安排反馈到下游产出决策;反向地,下游需求的收敛也会通过数量与预期渠道压缩上游投资、产能利用与经营性现金流。在这一“实物流—信息—信用”交织的网络中,政策冲击很少即刻、对称地体现在所有链路上,而是呈现出“方向性、阶段性与选择性”三种特征。

方向性在于,行业配对(i→j)所承载的经济含义并不等同于(j→i):前者对应发送行业对接收行业的当期影响强度,后者则反映反向约束或需求侧反馈。若使用行业整体均值或无向相关,容易掩盖“谁影响谁”的真实结构,从而在政策评估中把“共同趋势”误判为“净效应”。阶段性在于,合同、库存与账期等摩擦普遍存在,政策当年的可观测效应常常并不显著,待到价格—数量—信用三通道逐步完成“再均衡”后,政策影响才在次年及其后持续显现。选择性在于,并非所有链路都以同样力度响应:对“处理邻居”的耦合强度更高的链路,可能在政策后体现出更强的抑制,这既与传播路径长度、耦合强度与议价结构相关,也与链路上“信息与协同”的可行性相关。

为刻画上述结构差异,本文在行业配对层使用方向性强度指标(II)作为被解释变量:II 由公司层面的有向预测关系汇聚而来,归一化后反映当期 i→j 的影响强度;并在政策前窗口构造“对处理邻居的预暴露”,用于识别“哪里更易被冲击、哪里更可能吸收冲击”的结构性异质性。直观地讲,预暴露越高,链路越紧密,越可能在冲击发生时依靠“议价—信息—协同”三类安排内生出缓冲机制:通过账期协调、阶段性价格稳定与成本分摊、联合排产与库存协同、联合采购与共享信用支持等工具,降低短期外溢并把冲击的一部分“消化”在链路内部。这一机制在“相对净效应”的框架下,表现为政策后 II 的相对下降更为明显。

上述框架在经验上导向两个可检验的命题:第一,处理组行业相对对照组的方向性强度 II 在政策后呈“相对抑制”(平均效应);第二,政策前对处理邻居的预暴露越高,政策后 II 的相对下降越明显(结构性异质性)。二者共同构成本文“方向性—因果识别—结构剂量”的识别主线。值得强调的是,本文的目标并非描述绝对的前后变化,而是在控制共同时间冲击与链路不变异质性的前提下,识别“政策如何改变联动”。

\section{研究假设}
基于上述理论框架,本文提出以下两个可检验假设,并明确其统计含义与验证路径:

\noindent\textbf{H1(平均效应:相对抑制)}:政策实施后,处理组行业相对对照组的方向性强度(II)增幅更小或出现下降,即“相对抑制”成立。

\noindent\textit{验证路径}:在行业配对层估计双重差分模型,设置配对固定效应与时间固定效应,并在配对层聚类稳健标准误。若交互项系数为负且显著,则表明政策对跨行业传播的相对抑制成立。为检验识别假设,采用事件研究在政策前窗口观察各相对期系数是否围绕零波动;在政策后窗口,系数应在合理的时滞后逐步显现并保持方向一致。若交互项不显著或为正,则需复核平行趋势、潜在共时干扰与口径设定,谨慎解释“无明显相对抑制”的可能性。

\noindent\textbf{H2(结构性异质性:剂量—响应)}:政策前对处理邻居预暴露越高的链路,政策后的相对抑制越强。

\noindent\textit{验证路径}:将样本按政策前预暴露的顶/底分位划分为“高/低暴露组”,分别在子样本上估计双重差分模型;若高暴露组的交互项估计值更为负向(且显著),则支持“剂量—响应”的结构性异质性。亦可通过在全样本中加入“预暴露×政策后”或“预暴露×处理×政策后”的交互项,检验交互系数是否显著为负。为保证分组内识别的有效性,应在政策前窗口对各组的相对期系数进行预检,并以阈值/聚合口径的敏感性扫描验证方向稳定性。若异质性证据不显著,可能源于阈值设置过于极端导致组内混合不足,或结构性梯度本身并不强,需要在稳健性中进一步检验。

上述两个假设共同回答“有没有相对抑制”与“在哪些链路更强”。它们面向经验上的可验证问题,均可通过双重差分与事件研究的常见语法落地。至于治理上的含义——例如将“预暴露”纳入前置监测清单、围绕关键链路开展差异化缓冲与协同披露——属于基于经验证据的政策讨论,并不构成本文的统计假设或识别对象。相关讨论将在结论与政策建议章节予以集中阐释。

\section{识别设定与变量}
本节说明“如何估计、看什么统计信号、如何解释”的实现要点,并与数据口径保持一致。首先,被解释变量为行业配对层的方向性强度 II,反映发送行业对接收行业在当期的影响强度;该指标由公司层面的有向预测关系聚合而得,并进行归一化以便跨配对比较。解释变量包括处理组指示、政策后指示及其交互项;控制变量在发送与接收两侧对称纳入,以缓解财务状态差异带来的遗漏偏误。模型在行业配对层设置个体固定效应以控制链路不变异质性,在时间层设置共同时间效应以吸收宏观冲击,标准误在配对层聚类以处理序列相关与组内相关。

其次,事件研究用于检验平行趋势与动态路径:以政策前最后一年为基准,构造相对期虚拟变量,观察政策前各期是否围绕零波动,并在政策后识别效应的显现与持续。若动态路径显示“政策当年效应不显著—次年起显现—随后持续/收敛”,则与合同、库存与账期等摩擦带来的时滞相一致,有助于从时间维度解释平均效应。

再次,结构性异质性采用“政策前对处理邻居的预暴露”作为剂量维度,仅用于分组或交互项检验,不作为工具变量。预暴露的阈值(如分位切分)、聚合方式(如两端最大值或均值)与方向口径(发送侧、接收侧或总量)可能影响识别的精度与组内混合度,因而在稳健性部分对关键阈值与聚合方式进行敏感性扫描。分组识别应在政策前窗口进行预检,以确保各组平行趋势的可接受性;在全样本交互项识别时,应报告交互系数的方向与显著性,并结合平均效应与动态路径统一解释。

最后,稳健性检验以“缩尾—滞后—窗口—安慰剂”的矩阵化设计覆盖关键口径风险:缩尾处理抑制极端值驱动;不同滞后设定检验II构造的鲁棒性;变动窗口(如剔除特殊年份)检验样本敏感性;安慰剂(虚假时点或虚假处理组)检验模型在“无政策”条件下的虚假显著概率。结果报告应以“平均效应—动态路径—结构异质性”的顺序组织叙事:首先呈现方向性强度的相对净效应,其次呈现政策前平行趋势与政策后时间结构,最后呈现基于预暴露的结构性梯度。通过这一顺序,可以把“方向性—因果识别—结构剂量”的主线清楚地传达给读者。

综上,本文在理论上以“价格—数量—信用”的三通道描述冲击在网络中的方向性与时滞,在识别上以双重差分与事件研究识别“相对抑制”与“动态路径”,并以政策前预暴露作为结构性剂量识别“哪里更强”。有关变量、样本与口径的更详细说明见第\ref{chapter:data_method}章;实证结果与异质性证据见第\ref{chap:empirical}章;政策含义与治理建议见第\ref{chap:conclusion}章。