% ============================================================================
% 第2章 文献综述(仅使用二级目录,不新增更深层级)
% ============================================================================
\chapter{文献综述及理论基础}
\label{chap:literature}

\section{概念界定}
为避免口径混淆,本文对核心概念做统一界定:
\begin{enumerate}
  \item 行业影响强度(\II):基于公司层格兰杰有向预测关系,按行业聚合得到“$i\to j$”在期 $t$ 的方向性强度,归一化至 $[0,1]$,用于进入 DID/事件研究识别相对净效应;
  \item 预暴露(Exposure):在政策前窗口(2010—2013年)对“处理邻居”的耦合强度聚合(如两端最大值/均值),作为结构性剂量用于异质性分组或交互;
  \item 识别框架:双重差分(DID)刻画平均处理效应,事件研究检验平行趋势与动态路径;固定效应设置为配对($i,j$)与时间($t$),标准误在配对层聚类。
\end{enumerate}

\section{文献综述(结构化)}
围绕“怎么量—为什么传—怎么识别”的主线,现有研究可分为三条相互补充的谱系:
\begin{enumerate}
  \item 溢出度量与网络连通性:从单资产流动性到系统连通性,度量从无向相关拓展到有向网络与方差分解,强调“向外/向内/净溢出”的区分与多层归因\citep{billio2012econometric,diebold2012better,diebold2014connectedness}。近年在中国语境下,连通性口径也被用于刻画结构调整与政策冲击下的行业联动与再配置,提示度量选择需与识别目标相匹配。本文不并行采用 VAR‑FEVD/TVP‑VAR/频域口径,而是选择可直接进入因果识别的链路级方向性指标(\II)。
  \item 生产网络与跨行业传播:微观冲击沿投入—产出关系传播,选择性放大形成宏观波动\citep{acemoglu2012network,carvalho2014micro}。中国背景下,产业数字化与链主治理等制度实践强化了产业链协同与信息流通道,改变了传播的方向与强度\citep{Zhao2023DigitalInputNetwork,Qu2025ChainMaster,Lin2022ChainLeaderSystem}。本文据此在链路层刻画“方向性—耦合强度”的结构异质性。
  \item 政策评估与识别设计:DID/事件研究在处理异期效应与平行趋势时需谨慎设置固定效应与聚类推断\citep{sun2021event,bertrand2004much,cameron2015practitioner,angrist2009mostly};在网络存在一般干扰时,分组/交互与安慰剂设计有助于降低混淆\citep{aronow2017interference,athey2018network}。在“去产能”等结构性政策场景,企业与行业层证据亦显示政策调整会通过多渠道影响绩效与再配置\citep{Wang2022DecapacityTFP}。本文采用统一口径(配对FE+时间FE、配对聚类SE)并以矩阵式稳健性(缩尾、滞后、窗口、安慰剂)验证结论稳定性。
\end{enumerate}

\section{理论基础}
围绕“方向性—动态—识别”三条主线,本文采用与经验证据紧密相连的理论支撑:
\begin{enumerate}
  \item 生产网络的方向性与选择性放大:投入—产出网络中,节点位置(上/下游)、路径长度与耦合强度决定冲击的方向性与选择性放大\citep{acemoglu2012network,carvalho2014micro}。当共同时间冲击被吸收后,链路级的相对变化(处理组相对对照组)可体现为“相对抑制”($\beta<0$),与网络乘数的差异化传导相一致;
  \item 动态调整与时滞的经验事实:合同刚性、库存与账期等摩擦常导致“政策当年效应不显著—次年起显现—随后持续/收敛”的时间结构。这种时滞属于经验层面的动态特征,解释了事件研究路径的形状;
  \item 网络情境下的因果识别与干扰:在异期处理与组间异质性存在时,DID/事件研究需要避免系数污染,采用恰当的固定效应与聚类推断\citep{sun2021event,bertrand2004much,cameron2015practitioner,angrist2009mostly};在一般干扰(interference)不可完全排除的网络环境下,可通过固定结构性剂量(预暴露)进行分组/交互检验,并辅以安慰剂与口径敏感性以验证稳健性\citep{aronow2017interference,athey2018network}。
\end{enumerate}

上述三点与本文的实证设计一一对应:方向性 \II 支撑 H1 的相对净效应识别,时滞解释事件路径形状,预暴露作为结构性剂量支撑 H2 的“剂量—响应”。
\paragraph{流动性与溢出测度}
度量“谁影响谁、影响有多强以及方向为何”的首要问题,是建立具有解释力与可复现实证口径的溢出(connectedness)或联动(spillover)指标。早期以流动性为核心的研究(如Amihud测度)强调单资产随时间的摩擦与补偿关系,但难以揭示主体间有向作用。连通性框架的兴起(如\citep{billio2012econometric,diebold2012better,diebold2014connectedness})将方差分解与网络表示结合,把“来自他人的解释份额”解释为“被他人影响的强度”,并进一步区分“向外/向内/净溢出”。这一谱系的优势在于:一是提供了“系统—子系统—主体”多层的归因;二是天然适配图结构,使“方向性”与“重要边”可以被识别。然而,其方法实现面临三个常见抉择:窗口滚动 vs. 时变参数(TVP)、时域 vs. 频域、参数稳健性 vs. 解释力。

就方法定位而言,本文不并行采用 VAR‑FEVD、TVP‑VAR 或频域连通性等系统级度量;相关方法仅作为文献对话中的背景与参照,用以说明不同口径的优劣取舍与适用场景,避免读者误解为本文所使用的识别与度量口径。

基于上述权衡,本文采取“公司层格兰杰显著性→行业配对归一化”的方向性强度度量(II):先在公司层识别显著的有向预测关系,再按行业聚合为“i→j”在当期的显著性占比,并归一化到[0,1]。这一口径有三点考虑:(1)链路粒度:面向“配对(pair)/链路”而非“行业整体”,可以直接进入双重差分(DID)识别相对净效应;(2)方向语义:II天然表达“谁到谁”的当期方向性,避免无向相关性掩盖非对称传播;(3)稳健性矩阵:II对滞后阶、阈值与极端值处理敏感,需配套“滞后(lag=1/2/3)、缩尾(none/1/5/10%)、窗口(去极端年)、安慰剂(虚假时点/组)”等矩阵化稳健性。与此同时,我们承认FEVD/TVP‑VAR/频域等方法在描述“系统级—周期段”的解释力,因而在文献对话中对其优缺点进行“可替代但不冗余”的定位:II强调“链路级—当期—可入DID”的可识别性,系统级刻画可以作为补充,但不宜在本文中与主识别口径并行,以免产生口径杂糅与推断混淆。

方法学上,“方向性—有向网络”是连通性框架区别于一般协方差或相关系数的根本所在。Diebold–Yılmaz的方差分解、Baruník–Křehlík的频域分解与TVP‑VAR等方法各有侧重,但在本文中均不作为实证口径使用,仅作背景参照。本文从识别设计出发,将 II 作为“进入因果识别”的主口径:一方面在图结构上具备“边”的含义,另一方面在估计上可以与事件研究、DID与预趋势检验无缝衔接。当核心命题从“是否存在联动”转向“政策如何改变联动”时,度量的可识别性优先于系统层全面性,II 作为主口径更契合本文的目标函数。

\noindent\textit{与本研究假设的对应关系}:本节的方法讨论支撑 H1(平均效应)的可检验性——方向性 II 能以链路(i→j)为单位进入 DID/事件研究,从而识别“相对抑制”的符号与幅度;同时为 H2(结构性异质性)提供方向性语义与度量基础(预暴露按“对处理邻居”的方向性聚合),并为后文的政策启示留出工具化空间。
\paragraph{行业间风险传染与产业链结构}
生产网络文献强调:微观冲击如何沿投入—产出关系传播,并在某些节点被放大,最终形成宏观层面的波动\citep{acemoglu2012network,carvalho2014micro}。在这一框架下,“节点—边—路径—团簇”的组合结构决定了传播的幅度与方向性;若把行业视为节点、交易视为边,则上游供给约束与下游需求收缩会沿既有链路累积,形成“瓶颈—溢出—放大”的链式逻辑。中国背景下的相关研究显示:生产网络结构与阶段性政策共同作用,数字化投入强化了产业内外部关联\citep{Zhao2023DigitalInputNetwork}。同时,行业的“上行/下行风险”并不对称——上行情绪可能局部提升“向外”强度,而下行期的库存/订单约束更快地压缩“向内”接收能力\citep{Li2024UpsideDownsideSpillover}。这些事实共同提示:仅在行业总体层面比较均值,容易忽略“谁影响谁”的非对称性;仅在宏观层面观察波动,也难以对链路级别的治理提出具体工具。

与生产网络并行,金融联动与市场预期也在“放大—吸收”的过程中发生作用:价格、数量与信用约束的三通道会出现阶段性“共振”或“抵消”,冲击在某些链路可能被扩张融资或合同协同所“吸收”;而在另一部分链路,则因融资边际收缩与信息不对称被显著加强。把这三通道纳入解释框架,一方面有助于识别“政策当年效应可能不显著—随后显现”的时滞结构,另一方面也提醒我们在因果识别中严谨控制共同时间冲击,避免把“共振”误判为“政策”。

本文采取“链路(pair)层”的视角,直接在“i→j”的方向上检验政策影响强度的相对净效应,避免行业整体层可能掩盖的非对称传播;同时,以“政策前对处理邻居的预暴露”刻画结构性耦合强度,识别“冲击更易传到哪里、在哪些链路更可能被吸收”的异质性。在此过程中,“事实—识别—机制”的分层叙事尤为重要:事实层面,政策前后网络结构是否发生可观测变化;识别层面,DID/事件研究是否支持“相对抑制”的因果陈述;机制层面,结构性耦合是否解释“抑制更强”的剂量—响应。

\noindent\textit{与本研究假设的对应关系}:本节的生产网络与传播逻辑为 H2(结构性异质性)提供理论支撑:预暴露越高,沿链路的耦合越强,越可能在政策后呈现“更强抑制”的剂量—响应;同时也为 H1(方向性联动的“相对抑制”)提供了“为何可能”的结构直觉。
\paragraph{政策评估的识别方法}
双重差分(DID)在政策评估中的长处在于:通过处理组与对照组在政策前后变动的差异,识别“相对净效应”,从而部分剔除共同趋势。但在分期施治与组间异质性广泛存在时,传统TWFE(双向固定效应)框架的“组—时”系数可能被异期效应污染,导致预趋势检验出现“伪显著”或政策后效应被低估/夸大\citep{sun2021event}。因此,本文在设计上把“基准DID+事件研究”作为识别主干:前者提供“平均处理效应(ATT)”的估计,后者用“相对期(k)”的路径系数检验平行趋势与时滞结构。从固定效应角度,我们在配对(pair)层设置个体固定效应控制不随时间变化的“链路异质性”,在时间层设置共同时间效应以吸收宏观冲击;标准误在配对层聚类,处理同一配对的序列相关与组内相关\citep{bertrand2004much,cameron2015practitioner}。

识别假设的操作化包括三点:其一,平行趋势——用事件研究在政策前窗口检验各相对期系数是否显著偏离0;其二,无共时重大政策干扰——在口径声明与稳健性设计中增加对潜在干扰年份/事件的处理(例如去极端年或替代窗口);其三,一般干扰(interference)——在网络环境下,链路间可能存在相互影响,本文通过在分组内固定“政策前预暴露”并作为结构性剂量进行异质性识别,减少链路间交叉溢出对“平均效应”的污染,并在稳健性中扫描阈值、聚合与方向口径以验证稳健结论\citep{aronow2017interference,athey2018network}。此外,为缓解口径敏感性,我们采用“缩尾—滞后—窗口—安慰剂”四维矩阵:缩尾处理(none/1/5/10%)抑制极端值的驱动效应;滞后(lag=1/2/3)验证II构造与预测结构的鲁棒性;窗口(例如剔除特殊年份)检验样本敏感性;安慰剂(虚假时点/虚假处理组)检验模型在“无政策”条件下的虚假显著概率。

\noindent\textit{与本研究假设的对应关系}:本节识别框架直接服务于 H1——DID估计量与事件研究的路径系数分别刻画“相对净效应”和“动态演化/平行趋势”;同时提供 H2 的实施语法(预暴露×Post 或分组 DID/事件对比)以验证“剂量—响应”。
在“度量—结构—识别”的框架中,DID/事件研究回答“政策有没有改变联动、方向与平均幅度如何”;而要解释“哪里更抑制/更显著”,需要引入结构性剂量维度。下一节我们讨论“预暴露”作为结构剂量的设计逻辑与机制含义。

\paragraph{产业链结构与稳定器机制}
“预暴露(exposure)”指政策前窗口内,行业相对于“处理行业集合”的结构性耦合强度。本文采用“向处理邻居的II强度聚合(如两端max)”作为链路层的结构剂量,用于异质性分组或交互,不作为工具变量以避免Bartik类识别争议\citep{goldsmithpinkham2020bartik}。其经济含义在于:预暴露越高,链路越紧密,越可能存在内生的“协同—分担—缓冲”安排,从而在政策后出现“II下降更显著”的抑制现象。

稳定器机制可以从三条渠道解释:第一,议价与合同协同——处理邻居在冲击期更倾向于协调账期、阶段性锁价、共同分担边际成本,通过“合同—信用—供货节奏”的组合协商来内部化短期冲击;第二,信息优势与联合排产——紧密链路更早共享供需与库存信息,形成“联合排产/备货—柔性产能—库存协同”,降低预测误差与被动减产所致的外溢;第三,共享资源与信用支持——在关键时点形成“联合采购与授信背书”安排,以抵消下游需求或上游供给的阶段性波动。三者共同作用,使高预暴露链路在政策后呈现更强的“相对抑制”,而低预暴露链路则更易体现“市场化传导”的直接外溢。

从设计角度,预暴露的阈值(分位数切分)、聚合(max/mean)与方向(out/in/total)口径会影响“组内混合度—噪声”之间的平衡。经验上,为兼顾异质性识别与预检(平行趋势)的稳定性,可取顶/底分位(例如q=0.30)进行“高/低暴露”分组,并丢弃中间样本以提高组内同质性;同时在附录报出敏感性扫描,展示在不同阈值与聚合口径下结论方向一致。需要强调的是:预暴露属于“结构性描述量”,目标是帮助解释“在哪里更抑制”,而非作为外生工具直接识别因果;因此,我们把它限定在“分组/交互—剂量—响应”的语法中与DID/事件研究配合使用。

\noindent\textit{与本研究假设的对应关系}:本节机制阐释与口径设计正对应 H2:若高预暴露链路在政策后更易通过议价/信息/协同内生出缓冲机制,则应观察到“高暴露组的DID更负/事件路径更向下”;该结构性证据亦可转化为“链路治理”的政策启示依据(不作为本文的可检验假设)。
\section{文献述评与研究空白}
综合上述脉络,可以把现有研究的“位势图”概括为三种偏好:一类聚焦“系统层的联动事实”,在连通性与网络拓扑上给出丰富描述,但对“为什么变化、政策如何改变联动”的因果阐释不足;一类聚焦“政策评估的平均处理效应”,在个体/行业层进行DID识别,但缺乏在网络情境下的方向性刻画与结构维度;还有一类使用“节点属性(强度/位置)”解释结构差异,但在“哪里更抑制”的问题上缺少可度量、可操作的剂量维度。相较之下,本文在三方面补位:(1)在网络情境下以“方向性II”进入DID与事件研究,识别“相对净效应”,避免将共同趋势误判为政策;(2)以“政策前对处理邻居的预暴露”作为结构性\emph{剂量},从“更强—更弱”的梯度关系来解释“哪里更抑制”,并通过阈值/聚合敏感性扫描降低经验性参数依赖;(3)在稳健性矩阵上对“缩尾—滞后—窗口—安慰剂”进行系统化覆盖,提高因果陈述的稳健程度。

在中国语境下,“去产能—稳链—反内卷”提供了一个将证据转译为治理工具的应用场景:本文的结构化识别为“链路治理”提供三条可操作路径——关键链路的\emph{前置校准}(早识别、早介入)、\emph{差异化缓冲}(账期/锁价/联合采购/信用支持)与\emph{协同披露}(跨部门与链路的信息同步)。这类工具并不依赖于某一行业或某一模型,而源自“方向性—结构剂量—相对净效应”的方法学闭环,具有跨行业的可迁移性。

\noindent\textit{与本研究假设的对应关系}:本节综述把文献位势图与本文三大假设对齐:H1——在方向性度量与DID/事件研究框架下识别“相对抑制”;H2——以“政策前对处理邻居的预暴露”刻画结构剂量并验证“剂量—响应”;并据此将 H1+H2 的经验证据转译为“链路治理”的可操作工具(前置校准、差异化缓冲、协同披露),作为本文的政策启示(非可检验假设)。
