% ============================================================================
% 第1章 绪论(按项目大纲,仅二级目录:1.1/1.2/1.3)
% ============================================================================
\chapter{绪论}
\label{chap:intro}

\section{研究背景与意义}
2014年起,去产能政策在钢铁、煤炭、建材与部分高能耗制造业持续推进,客观上重塑受影响行业(处理组)的成本结构与资产负债表,并可能通过上下游交易、信用约束与共同预期触发跨行业风险联动。若停留在行业或宏观均值层面,既难以回答“谁影响谁、何时影响、影响几何”,也难以剥离共同趋势与结构性净效应。因而需要在网络情境下,以可识别的因果框架刻画政策对跨行业风险溢出的相对净影响。

网络视角的必要性体现在三点:
\begin{enumerate}
  \item 级联与非对称:上游供给冲击沿“价格—数量—库存约束”传导,下游需求收缩反向影响投资与现金流,方向性显著;
  \item 度量语义:行业间“向外/向内/净效应”并不对称,无向相关易掩盖“谁影响谁”的本质;
  \item 融资与预期共振:金融约束与市场预期与实物流共同作用,使“同向共振—异向抵消—局部瓶颈”并存,需在“有向、随时间变化”的框架下识别路径与强度。
\end{enumerate}

本文采用“公司层格兰杰显著性→行业配对归一化”的方向性强度指标,记为 \II(取值归一化至 [0,1]),以“链路/配对($i\to j$)”为单位进入 DID 框架识别政策的\emph{相对净效应}。并将政策前窗口(2010—2013 年)中行业与“处理邻居”的结构性耦合强度定义为“预暴露”,用于刻画\emph{结构性异质性}:链路越紧密,越可能内生形成“议价—信息—协同”的稳定器,由此在政策后表现为更明显的抑制效应。

理论意义在于:在网络情境下提供方向性溢出的因果识别,将连通性/生产网络与 DID/事件研究耦合;将“预暴露”作为结构性剂量纳入链路层分析,补足“行业/公司”层之外的证据。现实意义在于:把“稳链—补链—强链”的宏观目标转译为可操作的\emph{链路治理}工具,包括\emph{前置校准}(将预暴露纳入监测清单)、\emph{差异化缓冲}(账期协调、阶段性价格稳定、联合采购与信用支持)与\emph{协同信息披露}(跨部门与链路的信息共享),减少“同质化扩张—价格战—再过剩”的内卷回路。



\section{研究内容和技术路线}
本文围绕三项可检验问题:
\begin{enumerate}
  \item 平均效应:去产能实施后,处理组相对对照组的 \II 是否下降?若 DID 交互项系数 $\beta<0$,则为相对抑制;
  \item 动态与识别:政策前是否满足平行趋势?政策效应沿时间如何演进(事件研究)?
  \item 结构性异质性:预暴露是否呈现“剂量—响应”,即高预暴露链路抑制更强?
\end{enumerate}

技术路线遵循“事实—识别—机制—稳健”。第一,事实:在政策前后窗口构建行业配对的风险溢出网络,报告方向性强度与结构特征;第二,识别:在统一口径下实施 DID 与事件研究,设置配对固定效应与时间固定效应,标准误在配对层聚类,确保相对净效应与动态路径的可解释性;第三,机制/结构:以“预暴露”刻画结构性异质性,采用分组或交互识别“剂量—响应”;第四,稳健:在缩尾、滞后、样本窗口与安慰剂等维度进行系统检验。

数据与口径声明(与第\ref{chapter:data_method}章一致):样本窗口 2010—2018 年(月度),行业口径 40 个,形成 1,599 个有向行业配对,共 148\,050 个“配对—月份”。政策时点 2014-01。主结果采用滞后 $=3$ 与 $5~\%$ 缩尾;其余滞后与缩尾口径仅作稳健性对照。变量构造:\II 反映当期方向性强度,由公司层格兰杰显著性占比汇聚并归一化得到。预暴露刻画政策前窗口对处理邻居的结构性耦合强度,仅用于分组与展示;控制变量在发送与接收两侧对称纳入(资产负债率、ROA、TobinQ),以减少遗漏异质性的风险。



\section{创新之处}
\begin{enumerate}
  \item 识别:在网络图的绝对对比之外,提供方向性 \II 的 DID/事件研究因果估计,避免共同时间趋势被误判为政策效应;
  \item 结构:以“对处理邻居预暴露”作为链路层结构剂量,验证“剂量—响应”,补足行业/公司层之外的结构性证据;
  \item 机制:提出“议价—信息—协同”的稳定器机制,并与链路治理工具对接(稳链优先序、协同披露、分层压力测试)。
\end{enumerate}

局限与边界(不影响主结论):(1)口径依赖——主结果使用滞后 $=3$ 与 $5~\%$ 缩尾,其他设定仅作稳健性对照;(2)机制变量的直接可观测性有限——价格/账期/信用链条等“软变量”在数据可得时可作为后续扩展;(3)统一政策时点便于识别平均效应,但分行业/分地区节奏的细目刻画仍有拓展空间。
