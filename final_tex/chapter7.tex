% ============================================================================
% 第X章 研究结论与政策建议(顺序由 main.tex 决定,此处将作为第七章)
% ============================================================================
\chapter{研究结论与政策建议}
\label{chap:conclusion}

\section{主要结论}
基于2010–2018年40个行业、1,599个行业配对的月度面板数据,本文在行业配对层采用双重差分(DID)与双向固定效应、配对层聚类稳健标准误的统一设定,识别供给侧结构性去产能政策对跨行业风险溢出的影响。主要结论如下:
\begin{enumerate}
  \item \textbf{相对抑制的平均效应}:在控制共同时间趋势后,处理组行业相对于对照组的行业影响强度($II$)显著下降,DID系数显著为负,表明政策对跨行业风险外溢具有抑制作用(见第\ref{sec:baseline_did}节)。
  \item \textbf{动态路径与可检验前提}:事件研究显示政策前各期系数不显著,支持平行趋势假设;政策实施后效应逐步显现并持续存在(见第\ref{sec:parallel_trends}节)。
  \item \textbf{结构性证据(预暴露梯度)}:以“对处理邻居的政策前预暴露”对链路进行分组,预暴露越高的链路在政策后$II$降幅更大,可由“产业链稳定器”机制(议价、信息、协同)加以解释(见第\ref{sec:network_analysis}节及其异质性小节)。
  \item \textbf{稳健性}:在控制变量扩展(加入资产规模)与极端值处理(不同缩尾阈值)两个维度下,主结论的方向与显著性保持稳定(见第\ref{chap:robustness}章)。
\end{enumerate}

综上,本文在“网络—因果识别”的统一框架下,提供结构性产业政策影响跨行业风险传导的链路级证据:政策不仅改变受处置行业的内部状态,也改变行业之间风险传播的相对强度。

\section{政策建议}
\subsection{监管者视角}
\begin{itemize}
  \item \textbf{稳链优先序}:将“对处理邻居预暴露”纳入链路级监测,优先稳住高预暴露链路,降低局部冲击在网络中的放大概率与幅度。
  \item \textbf{协同监管与信息披露}:在政策密集推进期,围绕关键链路建立跨部门的应急信息披露与沟通机制,减少信息不对称,提高政策传导效率。
  \item \textbf{压力测试与资源配置}:在宏观审慎压力测试中引入链路级情景,按预暴露度分层制定流动性安排与信用缓冲,优化监管资源配置。
\end{itemize}

\subsection{产业与金融机构视角}
\begin{itemize}
  \item \textbf{中长期配套}:对高预暴露链路提供阶段性价格稳定与账期协商,避免现金流错配沿供应链放大。
  \item \textbf{组合管理}:在行业配置与风险对冲中纳入链路级风险评估,利用“稳定器”链路降低投资组合的系统性暴露。
\end{itemize}

\subsection{面向“反内卷”的治理启示}
基于本文识别到的\emph{相对抑制}与\emph{预暴露梯度}事实,可将“反内卷”的治理目标具体化为可操作的链路治理工具。需要强调的是,下述建议属于基于识别事实的制度性启示,本文并未直接检验“反内卷”的结果成效。
\begin{itemize}
  \item \textbf{从同质化扩张转向链路治理}:将治理单元下沉到行业\emph{配对/链路},以高预暴露链路为优先稳链对象,减少“规模扩张—价格战—再过剩”的低质量竞争回路。
  \item \textbf{差异化缓冲与协同工具}:围绕关键链路前置安排差异化的信用与流动性缓冲(账期协调、联合采购、阶段性锁价)及跨部门信息共享,抑制级联放大。
  \item \textbf{前置校准的监管框架}:将预暴露指标纳入事前评估与日常监测,形成“识别—托底—协同”的闭环,提高网络层面的抗冲击能力与政策到位效率。
\end{itemize}

\section{局限与展望}
\begin{itemize}
  \item \textbf{度量口径}:$II$ 指标基于公司层格兰杰显著性聚合,受缩尾与聚合规则影响;后续可与其他连通性度量并联对比。
  \item \textbf{政策多点推进}:统一政策时点便于识别平均效应,但不同细目与地区节奏可能导致异质性,后续可细化到次级政策与区域层面。
  \item \textbf{机制识别}:本文将预暴露用于结构性分组与证据锚定,不作为强外生解释;未来在数据可得前提下,可结合关系契约、供应链金融等直接指标做机制验证。
\end{itemize}
